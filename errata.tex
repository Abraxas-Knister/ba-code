\documentclass[10pt,oneside]{scrartcl}
\usepackage[utf8]{inputenc}
\usepackage[T1]{fontenc}
\usepackage[ngerman]{babel}
\usepackage[top=3.5cm,bottom=3.5cm,left=3cm,right=3cm]{geometry}
\renewcommand{\labelitemi}{$\rightarrow$}
\newcommand{\timo}{$[..]$}
\newcommand{\etic}[2][\timo]{\glqq{}#1{}#2{}#1{}\grqq{}}
\newcommand{\etico}[1]{\glqq{}#1{}\grqq{}}
\setlength{\parskip}{0pt}
\pagestyle{empty}
\begin{document}
\textsl{Errata zur Bachelorarbeit \glqq Reversibilitätsuntersuchung in Quantensystemen mit Hilfe exakter
Diagonalisierung\grqq, Thomas Axmann:}
\begin{itemize}\itemsep0pt
%\item[S. 8,] abgesetzte Formel, ergänze fehlendes Komma hinter der Gleichung.
%\item[S. 11,] dritte abgesetzte Formel, ergänze fehlendes Komma hinter der Gleichung.
\item[S. 20,] Beweis, letzte Zeile: ersetze $n\!+\!1$ durch $n$ in \etico{\timo{} ergibt sich der Schritt auf
  $n+1$.}.
\item[S. 28,] zweite abgesetzte Formel, zweite Zeile: ersetze \etico{$\|w(t).\|$} durch \etico{$\|w(t)\|.$} .
\item[S. 36,] zweiter Absatz nach 5.2.2, nach erstem Doppelpunkt, ergänze: \etico{Es [!\,ist] deutlich ersichtlich} 
\item[S. 36,] abgesetzte Formel, zweite Zeile: ersetze \etico{$O_{\alpha\alpha'}$} durch
  \etico{$O_{\alpha\alpha}$}.
%\item[S. 36,] vorletztes Satz: korrigiere \etico{Der [!\,z]eitliche Mittelwert}.
%\item[S. 36,] letzte Zeile: korrigiere \etico{gewä[!\,h]lt}.
%\item[S. 38,] dritter Absatz, vierter Satz: korrigiere \etico{wohlgeord[!\,ne]ten}.
\item[S. 38,] dritter Absatz, vorletzter Satz: Reformuliere, um zu präzisieren inwiefern die Anordnung
  \glqq{}wesentlich [anders]\grqq{} ist, zu \glqq{}Das stark unterschiedliche Verhalten mag darin begründet
    sein, dass der Ne\'el-Zustand eine wesentlich stärkere Durchmischung modelliert als der hier
    vorgeschlagene Anfangszustand.\grqq{}.  
\end{itemize}
\vfill
\textbf{\large\sffamily Stand 30.09.2018\hfill gez. Thomas Axmann}
\end{document}
